

A major advantage of anonymous usage data is the fact that it captures developers in their natural habitat, without any observational bias. Deriving conclusions from hours of developers' field work is naturally more convincing than from hour-long, in-lab user studies. One type of questions that usage data is well-suited to answer uses measurement of the magnitude of occurence of a specific event. For instance, researchers may want to know ``How often do developers invoke the pull-up refactoring'' or ``How often is the file search invoked?''. By performing a count of a specific message in the collected logs, researchers can easily calculate frequencies of specific actions that can be often sufficient to answer important research questions. However, researchers must be wary of a few common issues with magnitude analysis. First, in any sufficiently large set of user logs there is a small set of users that will use the feature/tool under analysis orders of magnitude more often than the general population, potentially skewing the data. Second, any fine-grained attempt to qualify the raw counts requires making possibly incorrect assumptions about the data. For instance, there is a temptation to report refactorings per hour, yet any fine-grained time calculation requires assumptions about how time was spent between activities, which experience has taught us are often wrong. Note that coarse-grained qualification, such as refactorings performed per day, are possible.   
