

A major advantage of anonymous usage data is its sheer size. Deriving conclusions from thousands of hours of developers' field work is naturally more convincing than from hour-long, in-lab user studies. One of the types of questions that this data is well-suited to answer is questions about magnitude. Researchers might want to know ``How often do developers invoke the pull-up refactoring'' or ``How often is the file search invoked?''. By performing a scan of the collected logs researchers can easily calculate raw counts of these actions, yet they must be wary of a few common issues with these counts. First, in any sufficiently large set of user logs there is a small set of users that will use the feature/tool under analysis orders of magnitude more often than the general population, potentially skewing the data. Secondly, any fine-grained attempt to qualify the raw counts requires making possibly incorrect assumptions about the data. For instance, there is a temptation to report searchers per hour, yet any fine-grained time calculation requires assumptions about how time was spent between activities, which experience has taught us are often wrong. Note that coarse-grained qualification, such as searches performed per day, are possible.   
