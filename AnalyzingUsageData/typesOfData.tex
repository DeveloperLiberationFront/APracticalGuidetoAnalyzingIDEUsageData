\noindent
{\bf Non-anonymous data}, where sensitive information including source code snippets, change-sets, and even access to the source code base is provided has obvious advantages. Researchers can replay developers' activity stream, affording them a deep understanding of the developer's actions~\cite{UseDisuseMisuseRefactoringsExtendedVersion}. There are few limits on how this data can be analyzed, and {\em non-anonymous data is well-suited for exploratory studies}. Unfortunately, there are some key disadvantages. First, only developers working on open source systems are likely to participate in such studies; typical enterprise developers may face termination if they were to leak even parts of their source code base. Second, while playback and other deep analyses are possible, these analyses can be costly in terms of time and resources. 

\vspace{0.1in}

\noindent
{\bf Anonymous data}, where only records of activities and anonymous facts about artifacts are recorded, may at first seem strictly inferior. Indeed there are some limitations on what can be learned from anonymous activity streams, yet there are key advantages. First, developers are receptive to data collection for research purposes, and thus the ability to collect a large amount of information from many developers increases greatly. Second, because the data set is relatively large and is harvested from working developers, conclusions are ultimately more reliable. 

In this section we focus on analyzing anonymous data sources. We do so because analyzing anonymous activity streams is similar to analyzing non-anonymous data streams (i.e., they are both activity streams) and because the unlimited variation of analysis permitted by non-anonymous data affords few generalities. As we discuss analyzing usage data we start with straightforward magnitude analysis, build to a categorization system for activity streams, discuss dividing streams into sessions, and finally state-based analysis. 
