\noindent
{\bf Non-anonymous data} (TODO: is there a better term for this?!), where sensitive information including source code snippets, change-sets, and even access to the source code base, provides obvious advantages. Researchers can replay developers' activity stream, affording them a deep understanding of the developer's actions~\cite{UseDisuseMisuseRefactoringsExtendedVersion}. There are few limits on how this data can be analyzed, and {\bf non-anonymous data is well-suited for exploratory studies}. Unfortunately, there are some key disadvantages. First, only developers working on open source systems are likely to participate; typical enterprise developers may face termination if they were to leak even parts of their source code base. Second, while playback is now possible, analyzing this data can be costly in terms of time and resources. 

\vspace{0.1in}

\noindent
{\bf Anonymous data}, where only records of activities and anonymous facts about artifacts are recorded, may at first seem strictly inferior. Indeed there are some limitations around what can be inferred from anonymous activity streams. Yet the advantages make it a great complementary data source. First, developers are receptive to data collection for research purposes, and thus the ability to collect a large amount of information from many developers increases greatly. Second, because of this larger collection, while it may be more difficult to analyze anonymous data, any conclusions made on a larger data set collected from working developers are ultimately more believable, as they represent actual field usage. 

In this section we focus on analyzing anonymous data sources. We do so because analyzing anonymous activity streams is similar to analyzing non-anonymous data streams (i.e., they are both activity streams) and because the unlimited variation of analysis permitted by non-anonymous data affords few generalities. As we discuss analyzing usage data we start with straightforward magnitutde analysis, build to a categorization system for activity streams, and finally discuss dividing streams into sessions. 
