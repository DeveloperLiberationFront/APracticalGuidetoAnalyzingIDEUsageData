\vspace{0.1in}

\noindent
{\bf Anonymous data}, where only records of activities and anonymous facts about artifacts are recorded, may at first seem strictly inferior. Indeed there are some limitations around what can be inferred from anonymous activity streams. Yet the advantages make it a great complementary data source. First, developers are receptive to data collection for research purposes, and thus the ability to collect a large amount of information from many developers increases greatly. Second, because of this larger collection, while it may be more difficult to analyze anonymous data, any conclusions made on a larger data set collected from working developers are ultimately more believable, as they represent actual field usage. 

In this section we focus on analyzing anonymous data sources. We do so because analyzing anonymous activity streams is similar to analyzing non-anonymous data streams (i.e., they are both activity streams) and because the unlimited variation of analysis permitted by non-anonymous data affords few generalities. As we discuss analyzing usage data we start with straightforward magnitutde analysis, build to a categorization system for activity streams, and finally discuss dividing streams into sessions. 
