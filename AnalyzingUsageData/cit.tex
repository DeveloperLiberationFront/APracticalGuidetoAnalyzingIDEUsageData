The Critical Incident Technique (CIT) is a general methodology for improving a
process or system with respect to a set of objectives. CIT prescribes a
systematic study of the \emph{critical incidents}. Critical incidents are
\emph{positive} or \emph{negative} events that have a significant effect on the
objectives.

CIT was developed and published in its current form by Flanagan in
1954~\cite{Flanagan1954CIT}. Nevertheless, it is believed that the technique was
introduced even earlier by Galton (circa 1930). Variations of CIT have been
widely used in Human Factors~\cite{ShattuckWoods1994CIT}.

In 1986, del Galdo \etal applied CIT to Human-Computer Interaction (HCI) as part
of evaluating the documentation of a conferencing system~\cite{DelGaldo1986CIT}.
They asked the study participants to perform a task and report any incident that
they ran into. The researchers observed the participants during the study,
analyzed the reported incidents, and proposed improvements to the documentation
accordingly.

In the context of IDEs, Vakilian and Johnson adapted CIT to automated
refactorings~\cite{VakilianJohnson2014Alternate}. The goal of this study was to
identify the usability problems of automated refactorings by analyzing
refactoring usage data. The researchers found that certain events such as
cancellations, reported messages, and repeated invocations are likely indicators
of the usability problems of automated refactorings. By locating these events in
the usage data and analyzing their nearby events, the researchers were able to
identify 15 usability problems of the Eclipse refactoring tool. For instance,
the usage data indicated that six participants invoked the Move Instance Method
refactoring for a total of 16 times but none finished the refactoring
successfully. In all cases, the participants either canceled the refactoring or
could not proceed because of an error that the refactoring tool reported. By
inspecting these critical incidents, the researchers were able to infer two
usability problems related to Move Instance Method.

To apply CIT on a set of usage data, you should follow several steps. First,
identify the objectives. Finding usability problems is only one example.
Second, identify a set of events that may be critical incidents. These are
events they may have significant effects on your objectives. Usually, the
negative critical incidents, which may have negative effects on the objectives,
are better indicators of problems than the positive ones. Third, identify the
critical incidents in your usage data. Fourth, collect sufficient contextual
information to interpret the critical incidents. This may include events that
have occurred in close time proximity to the critical incidents. You may even
have to interview the users or ask them to report their explanations of the
incidents during the study. Although the user reports may be short or
incomplete, they can provide more insight about the nature of the problems.
Finally, evaluate the effectiveness of the critical incidents and revisit the
steps above. The process that we described for applying CIT on usage data is
iterative in nature. That is, it is natural to go back to a previous step from
each step.

