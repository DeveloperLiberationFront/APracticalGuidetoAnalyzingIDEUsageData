
Other data sources, such as developer feedback, task descriptions, change histories can provide context for usage data and yield new opportunities for supporting developers. In this section, we briefly outline some related work in this area.

\subsubsection{Developer Feedback}

Usage data collection can be augmented by explicit developer feedback to establish a ground truth. For instance, if you want to understand a developer's opinion of a tool or his or her purpose for using an IDE feature, it is best to ask about it. A post-analysis interview can shed light on your observations and confirm your analysis or point out different aspects.

Information on a developer's activities can provide an additional rational to the usage data and explain the developer's actions. Asking questions on how well the developer knows the code he or she is working on can explain a developer's navigation patterns and usage of code search~\cite{SnipesExperiencesGamifyingSoftwareDevelopment}, while questions on the task, e.g. implementing a new feature versus fixing a bug, can explain other characteristics, such as the amount of editing versus testing.

\subsubsection{Tasks}

As mentioned above, Mylin is a popular extension to the Eclipse IDE that supports task
management, reducing the effort for a developer to switch between
tasks and maintaining relevant information for each
task~\cite{Kersten-Mylin}. In the context of a specific task, Mylin
collects usage data in order to compute a degree-of-interest (DOI)
value for each program element, which represents the interest the
developer has in the program element for the task at hand. Program elements that a developer interacted with frequently and recently have a higher DOI value. Using these calculated DOI values, Mylyn highlights the elements relevant for the current task and filters unnecessary information from common views in the IDE. 

\subsubsection{Change History}

The FastDash tool enables real-time awareness of other developers'
actions (e.g. focus or edits to a specific file) by utilizing usage
data in concert with the source files and
directories~\cite{FastDash}. FastDash's purpose is to reduce faults
caused by lack of communication and lack of awareness of activities
that other developers are performing on the same code base. The tool
highlights other developer's activity as it is occurring using a
sophisticated dashboard on the developer's screen or on a team's dashboard.

