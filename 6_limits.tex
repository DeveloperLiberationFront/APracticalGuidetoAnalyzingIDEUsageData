\section{Limits of What You Can Learn from Usage Data}\label{sec:limitations}

Collecting usage data is potentially a boon to research, and as we have
pointed out, has many interesting and impactful applications.
Nonetheless, there are limits to what you can learn as a researcher
from usage data.
In fact, our experience is that researchers (and practitioners) have 
high expectations about what they can learn from usage data, and those
expecations often come crashing down after significant effort implementing
and deploying a data collection sytem.
So before you begin your usage data collection and analysis, consider
the following two limitations of usage data.

\textbf{Rationale is Hard to Capture.}
Usage data tells you what a software developer did, but not
why she did it.
For example, if your usage data tells you that a developer used 
new refactoring tool for the first time, from a trace alone you cannot determine whether
(a) she learned about the tool for the first time, (b) she had used the tool before,
but before you started collecting data, or (c) her finger slipped and 
she pressed a hotkey by accident.
A researcher may be able to distinguish between these by collecting additional information,
like asking the developer just after she used the tool why she used it,
but it is impossible to definitively separate these based on the 
usage data alone.

\textbf{Usage Data Never Captures Everything.}
Researchers often want to capture ``everything,'' or at least 
everything that a developer does.
This is fundamentally impossible, and the reason we introduced the 
goal-question metric.
If you have a system that captures all key presses, you are still 
lacking information about mouse movements.
If you have a system that also captures mouse movements, you're still
missing the files that the developer is working with.
If your system captures also the files, you still lacking the 
histories of those files.
And so on.
Ultimately, usage data is all about fitness for purpose -- is the data
you are analyzing fit for the purpose of your research questions.

To avoid these limitations, we recommend not thinking about 
usage data collection abstractly, but thinking about it concretely before
you begin.
Invent an ideal usage data trace, and ask yourself:

\begin{itemize}
  \item Does the data support my hypothesis?
  \item Are there alternative hypotheses that the data would support?
  \item Can the data be feasibly generated?
\end{itemize}

\noindent
Answering these questions will help you determine whether you can sidestep
the limitations of collecting and analyzing usage data.