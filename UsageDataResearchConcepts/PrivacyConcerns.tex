It is great to see how usage data may benefit your research, however, there are some privacy concerns your users might and often have regarding the data collection and who the data is shared with. These privacy concerns arise mainly since the collected data may expose individual users or it may expose parts of the source code companies are working on.

%Concerns over privacy makes usage monitoring a flash point for some people.  Privacy concerns can be divided into concerns about what data is collected that may expose the user or parts of the source code to prying eyes and who the data and aggregates of it are shared with.

To minimize privacy concerns about the collected data, steps such as encrypting sensitive pieces of information , for instance, by using a one-way hash-function can be taken. Hashing sensitive names, such as user names, window titles, filenames or source code identifiers, provides a way to obfuscate the data and reducing the risk of information that allows to identify the user or the projects and code they are working on. While this obfuscation makes it more difficult to analyze the exact practices, using a one-way hash-function will still allow to differentiate between distinct users, even if anonymous.

%To alleviate concerns about what data is collected, steps such as hashing sensitive pieces of information can reduce the concern. If you collect information like window titles they can contain filenames, web site titles, or even email titles.  
%Hashing these names provides a measure of assurance that the data are less easily identifiable with the user of the system and with sensitive subjects they may be working on.   
%Collecting a user identification makes this data very helpful for understanding distribution of practices, however, this makes it more sensitive as well.  Generating an anonymous identifier to identify distinct users helps alleviate the concern that data will be associated with the user.

Furthermore, to reduce concerns about who the data is being shared with, you should create an explicit policy. If the data is for your own research purposes only, you can specifically state who will have access to the data and what they will do with it. Limiting statements such as not reporting data at the individual level helps to reduce a user's privacy concerns.
