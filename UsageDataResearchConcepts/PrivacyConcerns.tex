Concerns over privacy makes usage monitoring a flash point for some people.  Privacy concerns can be divided into concerns about what data is collected that may expose the user or parts of the source code to prying eyes and who the data and aggregates of it are shared with.

To alleviate concerns about what data is collected, steps such as hashing sensitive pieces of information can reduce the concern. If you collect information like window titles they can contain filenames, web site titles, or even email titles.  Hashing these names provides a measure of assurance that the data are less easily identifiable with the user of the system and with sensitive subjects they may be working on.   Collecting a user identification makes this data very helpful for understanding distribution of practices, however, this makes it more sensitive as well.  Generating an anonymous identifier to identify distinct users helps alleviate the concern that data will be associated with the user.

To reduce concerns about who the data are shared with, create a statement of policy around this.  If the data is for research purposes only, you can state specifically whom will have access to the data and what they will do with it.  Limiting statements such as not reporting data at the individual level helps this policy reduce concerns about who the data is shared with.
