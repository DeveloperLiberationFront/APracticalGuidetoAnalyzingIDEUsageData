It is great to see how usage data may benefit your research. However, there are some privacy concerns your developers might and often have regarding the data collection and who the data is shared with. These privacy concerns arise mainly since the collected data may expose individual developers or it may expose parts of the source code companies are working on.  How you handle information privacy in data collection affects what you can learn from the data during analysis (see section \ref{sec:dataAnonymity}).

To minimize privacy concerns about the collected data, steps such as encrypting sensitive pieces of information , for instance, by using a one-way hash-function can be taken. Hashing sensitive names, such as developer names, window titles, filenames or source code identifiers, provides a way to obfuscate the data and reduce the risk of information that allows identification of the developer or the projects and code they are working on. While this obfuscation makes it more difficult to analyze the exact practices, using a one-way hash-function will still allow differentiation between distinct developers, even if anonymous.

Maintaining developer privacy is important but, there may be parts of your research where you need the ground truth that confirms what you observe in the usage data.  Thus you may need to know who is contributing data so you can ask them questions that establish the ground truth. A policy statement helps participants and developers be more confident sharing information with you when they know they can be identified with the information.  The policy statement should specifically state ho will have access to the data and what they will do with it. Limiting statements such as not reporting data at the individual level helps to reduce a developer's privacy concerns.
