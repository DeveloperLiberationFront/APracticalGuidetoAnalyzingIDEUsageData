

Small studies that examine a variety of data can generate metrics that you can apply to data collected in bigger studies where the additional information might not be available. 
%    with larger data source that does not have augmented data.  
        For instance, Robillard et al. defined a metric on structured navigation in their observational study on how developers discover relevant code elements during maintenance~\cite{wbsnipes:Robillard2004How}. This metric can now be used in a larger industrial study setting in which structured navigation command usage is collected as usage data, even without the additional information Robillard et al. gathered for their study.
%    For example, the metric structured navigation defined by Robillard et al.  in their study on how developers discover relevant code elements in maintenance\cite{wbsnipes:Robillard2004How}, supports a larger study where we can observe structured navigation command use in an industrial setting.

Finally and most importantly, usage data may not be enough to definitively answer your research questions. While usage data tells what a developer is doing in the IDE, it usually leaves gaps in the story (see section \ref{sec:limitations}).  Augmenting usage data with additional data sources such as user feedback, task descriptions, and change histories (see section \ref{sec:IncludingOtherSources}) can fill in the details necessary to understanding user behavior.