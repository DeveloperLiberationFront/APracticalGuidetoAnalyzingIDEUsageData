
We refer to the data about the interactions of users with an IDE as the
\emph{IDE usage data} or simply \emph{usage data}.

% Who are the stakeholders in this domain?
%
% How do the stakeholders benefit from usage data?
Several stakeholders benefit from capturing and analyzing usage data. First, IDE
vendors leverage the data to get insight about ways to improve their product
based on how their users use the IDE in practice. Second, researchers both
develop usage data collectors and conduct rigorous experiments to (1)~make
broader contributions to our understanding of programmers' coding practices and
(2)~improve the state-of-the-art programming tools (\eg, debuggers and
refactoring tools). Finally, programmers benefit from the analysis conducted on
the usage data because these analyses lead to more effective IDEs that make
programmers more productive.

% What are some examples of usage data?
At a high level, an IDE can be modeled as a complex state machine. In this
model, a user performs an action at each step that moves the IDE from one state
to another.
%
% The following is related to the subsection on "Usage Data Never Captures
% Everything".
Perhaps a video recording of the IDE accompanied by the keyboard strokes and
mouse clicks of the user would provide a complete set of usage data.
%
While this usage data in multimedia format may be suitable for small lab
studies, it suffers from two major limitations. First, it is difficult to
automatically analyze the usage data in this format. Second, the storage and
collection of usage data in this format is inefficient.

% Give examples of usage data and their uses.
To enable experiments beyond small labs, researchers and IDE vendors have
developed various usage data collectors (\SecRef{SecHowToCollectData}).
Depending on the goals of the experiments, the usage data collector captures
data about a subset of the IDE's state machine.

An exemplar of a usage data collection and analysis project with wide adoption
in practice is the Mylar project. Mylar started as a research project that later
became part of Eclipse. The Mylar project exhibits both of the advantages of
research on analyzing usage data as mentioned above.

The Mylar team analyzed collected and analyzed usage data to provide empirical
evidence about the usage frequency of various features of
Eclipse~\cite{V:Murphy2006How}.
%
Mylar~\cite{V:Murphy2006How} was one of the first usage data collectors. It was
implemented as a plug-in for the Eclipse IDE. Mylar captured users' navigation
histories and their command invocations. For example, it recorded changes to
selections, views, perspectives as well as invocations of commands such as
delete, copy, and automated refactorings.
%
Later, Eclipse incorporated a system similar to Mylar, called the Eclipse Usage
Data Collector (UDC)~\cite{WebPageUDC}, as part of the Eclipse standard
distribution package for several years. UDC collected data from hundreds of
thousands of Eclipse users every month. To the best of our knowledge, the UDC
data set is the largest set of IDE usage data that is publicly
available~\cite{WebPageUDCArchive}. Several researchers have mined this large
data set to
%
(1)~gain insight about programmers'
practices~\cite{VakilianJohnson2014Alternate, VakilianETAL2013Compositional,
V:MurphyHill2012How} and
%
(2)~develop new tools that better fit programmers'
practices~\cite{MurphyHill2012Improving, VakilianETAL2013Compositional,
Kersten-Mylar2005}.

In addition to collecting usage data, Mylar provided new features to the Eclipse
IDE that leveraged the usage data. This part of Mylar later evolved into a
plug-in called Mylyn that comes with the official distribution of Eclipse. Mylyn
introduces the concept of a \emph{task context}. It first collects the users'
actions to infer the set of entities (\eg, files, class, and packages) that are
related to a task. Mylyn then analyzes the entities associated with the current
task to surface relevant information with less clutter in various features such
as outline, navigation, and auto-completion\Fix{Refer to other parts of the
chapter that discuss Mylar/Mylyn}.

Studies on automated refactorings are another example of projects that have
collected and analyzed usage data. Researchers have analyzed the Eclipse UDC
data set~\cite{V:MurphyHill2012How, VakilianETAL2013Compositional}, developed
custom usage data collectors~\cite{VakilianETAL2012UseDisuseMisuse}, and
conducted survey and field studies~\cite{V:MurphyHill2012How,
VakilianJohnson2014Alternate, NegaraETAL2013ManualRefactorings} to gain more
insight about programmers' use of the existing automated refactorings. Several
studies have found that programmers do not use the automated refactorings as
much as refactoring experts expect programmers to~\cite{V:MurphyHill2012How,
NegaraETAL2013ManualRefactorings}. This finding motivated researchers to study
the factors that lead to low adoption of automated
refactorings~\cite{VakilianETAL2012UseDisuseMisuse, V:MurphyHill2012How} and
propose novel techniques for improving the usability of automated
refactorings~\cite{V:MurphyHill2012How, MurphyHill2012Improving,
MurphyHill2008ExtractMethod, LeeETAL2013DragDrop, MurphyHillETAL2011Gestures,
GeETAL2012BeneFactor, FosterETAL2012WitchDoctor}.
%	-tool usage
%	-resource usage (file/method/etc)
%	-click streams
%
%need for analysis
%	-answers questions about how developers work
%	-example questions in the past e.g. refactoring
%	-who uses the data
%		-researchers
%		-toolsmiths
%		-users (?)
%
%benefits
%	-understanding that you don't get from lab experiments. more realistic data
%	-you can also use it in the lab, but screencast is also inefficent in storage and analysis
%		- can get people working on their own tasks and own code with more realistic data
%		- can use in lab to make analysis easier than screencast
%		- longer period of time

