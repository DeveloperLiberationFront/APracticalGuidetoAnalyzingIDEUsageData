
We refer to the data about the interactions of users with an IDE as the IDE's
usage data or simply \emph{usage data}.

% Who are the stakeholders in this domain?
%
% How do the stakeholders benefit from usage data?
Several stakeholders benefit from capturing and analyzing usage data. First, IDE
vendors leverage the data to get insight about ways to improve their product
based on how their users use the IDE in practice. Second, researchers both
develop usage data collectors and conduct rigorous experiments to (1)~make
broader contributions to our understanding of programmers' coding practices and
(2)~improve the state-of-the-art productivity programming tools (\eg, debuggers
and refactoring tools). Finally, programmers benefit from the analysis conducted
on the usage data because these analyses lead to more effective IDEs that make
programmers more productive.

% What are some examples of usage data?
Abstractly, an IDE can be modeled as a complex state machine. In this model, a
user performs an action at each step that moves the IDE from one state to
another.
%
% The following is related to the subsection on "Usage Data Never Captures
% Everything".
Perhaps a video recording of the IDE accompanied by the keyboard strokes and
mouse clicks of the user would provide a complete set of usage data.
%
While this usage data in multimedia format may be suitable for small lab
studies, it suffers from two major limitations. First, it is difficult to
automatically analyze the usage data in this format. Second, the storage and
collection of usage data in this format is inefficient.

% Give examples of usage data and their uses.
To enable experiments beyond small labs, researchers and IDE vendors have
developed various usage data collectors (\SecRef{SecHowToCollectData}).
Depending on the goals of the experiments, the usage data collector captures
data about a subset of the IDE's state machine.
%
Mylar~\cite{V:Murphy2006How} was one of the first usage data collectors. It was
implemented as a plug-in for the Eclipse IDE. Mylar captured users' navigation
histories and their command invocations. For example, Mylar recorded changes to
selections, views, perspectives as well as invocations of commands such as
delete, copy, and automated refactorings.

Later, Eclipse incorporated a system similar to Mylar, called the Eclipse Usage
Data Collector (UDC)\Fix{Cite the UDC page}, as part of the Eclipse standard
distribution package for several years. UDC collected data from hundreds of
thousands of Eclipse users every month. To the best of our knowledge, the UDC
data set is the largest set of IDE usage data\Fix{Cite the UDC data set}.
Several researchers have mined this large data set to analyze to 
%
(1)~gain insight about programmers'
practices~\cite{VakilianJohnson2014Alternate, VakilianETAL2013Compositional,
V:MurphyHill2012How} and
%
(2)~develop new tools that better fit programmers'
practices~\cite{MurphyHill2012Improving, VakilianETAL2013Compositional,
Kersten-Mylar2005}.

Perhaps a success story of the impact that collection and analysis of usage data
can have on practice is Eclipse Mylyn. Mylyn started as research project that
was initially called Mylar. Mylyn introduces the concept of a \emph{task
context} to IDEs.


	-tool usage
	-resource usage (file/method/etc)
	-click streams

need for analysis
	-answers questions about how developers work
	-example questions in the past e.g. refactoring
	-who uses the data
		-researchers
		-toolsmiths
		-users (?)

benefits
	-understanding that you don't get from lab experiments. more realistic data
	-you can also use it in the lab, but screencast is also inefficent in storage and analysis
		- can get people working on their own tasks and own code with more realistic data
		- can use in lab to make analysis easier than screencast
		- longer period of time
