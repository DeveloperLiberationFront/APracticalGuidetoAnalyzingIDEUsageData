\section{ How to Extract Attributes from Usage Data}


  \begin{enumerate}
  \item Creating meaningful classification of events
	\begin{enumerate}
	\item
	Activities people working on eg. testing/debugging/search/edit
	\item
	Establish a goal for categorization
	\item
	Create categories for the goal
	\item
	Deciding on categories when it gets hard to tell
	\item
	Counting time considerations, time between events, eliminating downtime/time away
	\end{enumerate}
  \item Defining useful and balanced sequences % (Will, Anil)
	\begin{enumerate}
	\item
	Define a goal for sequencing data
	\item
	Sequences aligned with time or time events.  Eg. events per day or events
	between periods of inactivity.
	\item
	Natural sequences such as events between builds, events between edits, events between check-ins.
	\item
	Creating sequences that occur along an event count boundary
	\item 
	Dealing with multi-event dependent sequences where the sequence may depend on detection of a cluster of events
	\item
	Semi-automated ways to detect tasks people working on e.g. fix bug 33 (mylin monitor) can we do it automatically?

  \item  Modeling sequences with state models %(Anil)
  Once the sequence data is collected, we have to convert the data into a meaningful representation. 
                \begin{enumerate}
                \item Convert sequence data to usage data : 
                \item Calculate usage probability
                \item Calculate transitional probability
                \item Visualization
                \item Data mining
                \end{enumerate}
	\end{enumerate}

  \end{enumerate}
