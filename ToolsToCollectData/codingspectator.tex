\subsection{\CodingSpectator}
\label{CodingSpectator}

\CodingSpectator~\cite{CodingSpectatorWebPage, VakilianETAL2011Richer,
  VakilianETAL2012UseDisuseMisuse, VakilianETAL2013Compositional} is
an extensible framework for collecting Eclipse usage data. Although
researchers at the University of Illinois at Urbana-Champaign
developed \CodingSpectator{} primarily for collecting detailed data
about the use of the Eclipse refactoring tool, it also provides a
reusable infrastructure for \emph{submitting usage data} from developers to
a central repository.  \CodingTracker~\cite{NegaraETAL2012Dangerous,
  NegaraETAL2013ManualRefactorings, CodingTrackerWebPage} is another
data collector developed at Illinois, which collects finer-grained IDE
actions while reusing the data submission infrastructure provided by
\CodingSpectator.

\subsubsection{Collected Data}

\CodingSpectator{} was designed for capturing detailed data about the use of
automated refactorings. It collects three kinds of refactoring events:
\Canceled, \Performed, and \Unavailable. If a programmer starts an automated
refactoring but quits it before it finishes, \CodingSpectator{} records a
\Canceled{} refactoring event. If a programmer applies an automated refactoring,
\CodingSpectator{} records a \Performed{} refactoring event. Finally, if
a programmer invokes an automated refactoring but the IDE refuses to start the
automated refactoring indicating that the refactoring is not applicable to the
selected program element, \CodingSpectator{} records an \Unavailable{}
refactoring event.

Eclipse creates a \emph{refactoring descriptor} object for each \Performed{}
refactoring events and serializes it in an XML file. \CodingSpectator{} saves
more data in Eclipse refactoring descriptors of \Performed{} refactorings. In
addition, it creates and serializes refactoring descriptors for \Canceled{} and
\Unavailable{} refactoring events. \CodingSpectator{} supports
\Use{NumberOfCodingSpectatorSupportedRefactorings} of the
\Use{NumberOfEclipseAutomatedRefactorings} automated refactorings that Eclipse
supports.

We show a concrete example of the data that \CodingSpectator{}
collects for an invocation of the automated Extract Method refactoring
in Eclipse, which extracts a piece of code into a new method. This
refactoring moves a selected piece of code into a new method and
replaces the selected code by an invocation to the new method. To use
the automated Extract Method refactoring, a programmer has to go
through multiple steps. First, the programmer selects a piece of code
(\FigRef{FigCodingSpectatorExtractMethodSelectionExample}). Second,
the programmer invokes the automated Extract Method and configures it
(\FigRef{FigCodingSpectatorExtractMethodConfigurationExample}). In
this case, the programmer sets the name of the new method. The
configuration page provides a number of other options including method
accessibility, the ordering and names of method parameters, and the
generation of method comments. Third, after configuring the
refactoring, the programmer hits the ``Preview'' button and the
automated refactoring reports the problems that the refactoring may
introduce (\FigRef{FigCodingSpectatorExtractMethodErrorExample}). In
this example, the automated refactoring complains that the selected
name of the new method conflicts with the name of an existing
method. Finally, the programmer decides to cancel the refactoring and
\CodingSpectator{} records a refactoring descriptor for this
\Canceled{} refactoring, as shown in
\FigRef{FigCodingSpectatorDescriptorExample}. The type of a
refactoring event (\ie, \Unavailable, \Canceled, and \Performed) can
be inferred from the directory in which the XML file containing the
refactoring descriptor resides.  \CodingSpectator{} captures the
following attributes for the canceled automated Extract Method
refactoring in the above example.

\begin{enumerate}

\item \texttt{captured-by-codingspectator}: indicates that \CodingSpectator{}
  created the refactoring descriptor.

\item \texttt{stamp}: a time-stamp recording when the refactoring event occurred

\item \texttt{code-snippet}, \texttt{selection},
  \texttt{selection-in-code-snippet}, \texttt{selection-text}: the location and
  contents of the selection that the programmer made before invoking the
  automated refactoring

\item \texttt{id}: the automated refactoring's identifier

\item \texttt{comment}, \texttt{description}, \texttt{comments},
  \texttt{destination}, \texttt{exceptions}, \texttt{flags}, \texttt{input},
  \texttt{name}, \texttt{visibility}: configuration options, \eg{} input
  elements, project, and settings that programmers can set to control the effect
  of the refactoring

\item \texttt{status}: any problems reported by the automated refactoring to the
  programmer

\item \texttt{navigation-history}: when the programmer pressed a button to
  navigate from one page of the refactoring wizard to another

\item \texttt{invoked-through-structured-selection},
  \texttt{invoked-by-quick-assist}: selection method (\eg{} structured or
  textual selection and whether the automated refactoring was invoked using
  Quick Assist

\end{enumerate}

\begin{figure}
%
\centering
%
\includegraphics[width=\textwidth]{codingspectator-extract-method-selection.png}
%
\caption{\label{FigCodingSpectatorExtractMethodSelectionExample}A programmer
selects a piece of code to extract into a new method. The selected code is part
of class \texttt{SingleCustomOperationRequestBuilder} from commit
\texttt{bdb1992} of the open-source Elasticsearch project
(\texttt{https://github.com/elasticsearch/elasticsearch}).}
%
\end{figure}

\begin{figure}
%
\centering
%
\includegraphics[width=0.5\textwidth]{codingspectator-extract-method-configuration.png}
%
\caption{\label{FigCodingSpectatorExtractMethodConfigurationExample}A programmer
configures an automated Extract Method refactoring by entering the desired name
of the new method.}
%
\end{figure}

\begin{figure}
%
\centering
%
\includegraphics[width=0.6\textwidth]{codingspectator-extract-method-error.png}
%
\caption{\label{FigCodingSpectatorExtractMethodErrorExample}The Extract Method
refactoring reports a name conflict problem to the programmer. The programmer
can either ignore the problem and continue the refactoring, go back to the
configuration page to provide a different name, or cancel the refactoring.}
%
\end{figure}

\begin{figure}
%
\centering
%
\includegraphics[width=\textwidth]{codingspectator-refactoring-xml.png}
%
\caption{\label{FigCodingSpectatorDescriptorExample}An example refactoring
descriptor recorded by \CodingSpectator.}
%
\end{figure}

\subsubsection{Deploying \CodingSpectator}

Deploying \CodingSpectator{} consists of two main steps:
%
\begin{inparaenum}[(1)]
%
\item setting up a Subversion repository and
%
\item setting up an Eclipse update site.
%
\end{inparaenum}

\paragraph{Setting Up a Subversion Repository}

\CodingSpectator{} regularly submits developers' data to a central Subversion
repository. To collect \CodingSpectator's data automatically, you need to set up
a Subversion repository and create accounts for your developers. To allow the developers
to submit their data to the Subversion repository, you should grant them
appropriate write accesses to the repository.

Using a Version Control System such as Subversion as the data repository has
several advantages:

\begin{enumerate}
%
\item Subversion makes all revisions of each file easily accessible. This makes
  troubleshooting easier for researchers.
%
\item For textual files, Subversion submits only the \emph{changes} made to the
  files as opposed to the entire new file. This differential data submission
  leads to faster submissions.
%
\item There are libraries such as SVNKit\footnote{http://svnkit.com/} that
  provide an API for Subversion operations such as add, update, remove, and
  commit. \CodingSpectator{} uses SVNKit for submitting developers' data to the
  central repository.
%
\item Setting up a Subversion server is a well-documented process. This avoids
  the burden of setting up a specialized server.
%
\end{enumerate}

On the other hand, a disadvantage of using Subversion as the data repository is
that it requires the developers to maintain a copy of their data on their file
systems. The Subversion working copy on the developers' systems takes \emph{space}
and can also cause \emph{merge conflicts}, \eg, if a developer restores the contents
of the file system to an earlier version. To handle merge conflicts,
\CodingSpectator{} has built-in support for automatic conflict detection and
resolution. When \CodingSpectator{} detects a merge conflict, it removes the
developer's data from the central repository and then submits the new data. Despite
removing the data from the central repository, the researchers can still locate
the merge conflicts and restore the data that was collected before the conflicts occurred.

\CodingSpectator{} prompts the developers for their Subversion user names and
passwords when \CodingSpectator{} is about to submit their data.
\CodingSpectator{} gives the developers the option to save their passwords in Eclipse
securely. See \url{http://codingspectator.cs.illinois.edu/documentation} for
more information about the features of \CodingSpectator{} for developers.

\paragraph{Setting Up an Eclipse Update Site}

Users of \CodingSpectator{} install it from an Eclipse update
site\footnote{\url{http://codingspectator.cs.illinois.edu/installation}}. An
Eclipse update site is an online repository of the JAR and configuration files
that Eclipse requires for installing a plug-in.

You will have to customize \CodingSpectator{} at least by specifying the URL of
the Subversion repository to which \CodingSpectator{} should submit developers' data.
You may also want to customize the message that \CodingSpectator{} shows to the
developers when it prompts them for their Subversion credentials. You can customize
these aspects of \CodingSpectator{} by changing the configuration files that are
packed in the existing JAR files hosted at the Eclipse update site of
\CodingSpectator. If you need to customize \CodingSpectator{} in more complex
ways that involve changes to its source code, you should follow the instructions
for building \CodingSpectator's update site from source code.

\subsubsection{Extending \CodingSpectator}

In addition to collecting detailed refactoring data, \CodingSpectator{} provides
a reusable infrastructure for collecting Eclipse usage data. Extending
\CodingSpectator{} frees researchers from having to develop many features from
scratch, \eg, Subversion communications, automatic merge conflict detection and
resolution, secure storage of Subversion credentials, and periodic update
reminders.

\CodingSpectator{} provides an Eclipse extension point (id =
\texttt{edu.\-illinois.\-codingspectator.\-monitor.\-core.\-submitter}) and the
following interface:

\begin{lstlisting}
public interface SubmitterListener {
  // hook before svn add
  void preSubmit();
  // hook after svn add and before svn commit
  void preCommit();
  // hook after svn commit
  void postSubmit(boolean succeeded);
}
\end{lstlisting}

The above interface provides three hooks to \CodingSpectator's submission
process. \CodingSpectator{} checks out the Subversion repository into a folder,
which we refer to as the \emph{watched folder}. Then, it executes the Subversion
commands (\eg, add and commit) on the watched folder. A plug-in that extends the
\Code{submitter} extension point and implements the \Code{SubmitterListener}
interface can perform actions before or after two of the Subversion commands that
\CodingSpectator{} executes: add and commit.
%
For example, \CodingSpectator{} overrides the method \Code{preSubmit} to copy
the recorded refactoring descriptors to the watched folder. As another example,
the developers of \CodingSpectator{} made the Eclipse UDC plug-in use the
\Code{submitter} extension point and copy the UDC data to the watched folder. As
a result, \CodingSpectator{} submits the UDC data to the Subversion repository.
Effectively, this is an alternative method to the one presented in
\SecRef{SecUDCHowToUseIt} for collecting UDC data in a central repository.

%TODO: The paragraph above feels like a wrapup partiularly the last sentence.  Can you craft one more sentence with final words of advice to those who want to use Coding Spectator for their research?
% LocalWords: CodingSpectator Urbana Champaign IDE timestamp
%
% LocalWords: CodingTracker refactoring refactorings refactoring's
%
% LocalWords: SVNKit URL usernames online API

