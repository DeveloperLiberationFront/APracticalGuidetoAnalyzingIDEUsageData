\documentclass{book}

% The following commands make LaTeX not consider the periods in a few
% abbreviations as periods at the end of sentences.
\newcommand*{\eg}{e.g.\@}
%
\newcommand*{\ie}{i.e.\@}
%
\makeatletter
%
\newcommand*{\etal}{\@ifnextchar{.}{et al}{et al.\@}}
%
\makeatother

% The following commands should be used to define key and value pairs.
\newcommand{\Def}[2]{\expandafter\newcommand\csname rmk-#1\endcsname{#2}}
%
\newcommand{\Use}[1]{\csname rmk-#1\endcsname}

% Wraps notes on what needs to get fixed.
\newcommand{\Fix}[1]{\textcolor{red}{[#1]}}

% Wraps junk text.
\newcommand{\Comment}[1]{}

% Wraps good text that needs to be excluded, e.g., due to space limits.
\newcommand{\Space}[1]{}


\usepackage{alltt}
\usepackage{pdfpages}
\usepackage{hyperref}
\begin{document}

\title{ASD}

\chapter{A Practical Guide to Analyzing IDE Usage Data\vspace{-0ex}}
\author{
Will Snipes, Emerson Murphy-Hill,
Thomas Fritz, Mohsen Vakilian \\
Kostadin Damevski,
Anil R. Nair, David Shepherd
}
\maketitle
\thispagestyle{empty}
\pagestyle{empty}

\section{Abstract}
As software development evolved, Integrated Development Environments (IDEs) sprang up to aid the developer in managing the ever increasing complexity of software programs.  Meant to increase productivity, modern IDEs such as Eclipse and Visual Studio provide tools and capabilities to perform tasks such as navigating among classes and methods, continuous compilation, code refactoring, automated testing, and integrated debugging.  The breadth of development activities supported by the IDE makes collecting editor, command, and tool usage data valuable for analyzing developers' work patterns.  

Prior work in the RSSE book on Collecting and processing data for recommendation systems provides a description of tools, analysis methods, and important findings from developer usage data analysis.  The RSSE chapter is also a great introduction to prior research applying usage data to solve issues that challenge developers.  In this chapter, we will provide a practical how-to description for collecting and analyzing usage data from IDEs.  Each topic will provide practical guidance with how-to examples and exercises.  We also provide a role-based view of analysis describing what a software engineering researcher can do with the data, what a developer can do, and what toolsmiths can learn from the data.

Instrumenting the IDE to collect usage data and observing developers’' work through ethnographic studies gives researchers a greater level of detail on developers’' work than previously possible. Instrumenting the IDE involves extending the IDE within the provided API framework.  Eclipse and Visual Studio support a rich API framework allowing logging of many commands and actions  as they occur.  Tools used for research that instrument the Visual Studio IDE include Hackystat\cite{V:johnson2003beyond}, Blaze, and Codealike.  Tools that instrument the Eclipse IDE include Mylyn Monitor, CodingSpectator, Hackystat\cite{V:johnson2003beyond}, and Zorro\cite{Kou2010Operational}.

Assembling IDE usage data across developers and temporally across many development sessions allows researchers to study differences and similarities in work patterns.   Raw usage data must be transformed to answer specific research questions.  Techniques to transform usage data include classification, attribute extraction, deriving sequences, state-based models, and  multi-event sequences.  

Usage data supports analysis of how developers spend their time\cite{V:johnson2003beyond}, what activities might benefit from greater tool support \cite{V:MurphyHill2012How}, where developers have difficulty comprehending code \cite{Carter2010Are}, even whether they are following specific practices such as test-driven development\cite{Kou2010Operational}.  Combining usage data with additional data dimensions such as change history or code analysis, researchers can understand larger influences of low-level developer practices.  With these data we can answer questions such as the level of expertise developers have for an area of source code. \cite{Fritz2010Degreeofknowledge}

Currently, few developers study their own work habits with such data.  However, in a world of devices such as fitbit \footnote{fitbit device to track physical activity www.fitbit.com}, we may see self-analysis play a more important role.  By studying their own usage patterns, developers may learn how to improve their own practices or help their peers advance.

There are limits, however, to what IDE usage data can tell us.  The missing elements include things like how well the developer knows the subject code base, what their mental model of the code is, or how they intend to alter the code to suit new requirements.  The developers'' experience, design ideas, and constraints they keep in mind during an implementation activity are factors that we must obtain separately.  

Looking forward, usage data from development environments provides a platform for greater understanding of developer's low-level practices.  We expect to uncover more nuggets of how developers create mental models, how they investigate code, how they perform mini trial and error experiments, and what might release further productivity improvements for everyone.


\section{What Is Usage Data and Why Analyze It?}

types of data
	-tool usage
	-resource usage (file/method/etc)
	-click streams

need for analysis
	-answers questions about how developers work
	-example questions
	-who
		-researchers
		-toolsmiths
		-users (?)

benefits
	-understanding that you don't get from lab experiments. more realistic data
	-you can also use it in the lab, but screencast is also inefficent in storage and analysis

%lstlisting settings for section 2
\lstset{language=Java,
captionpos=b,
tabsize=3,
frame=lines,
numbers=left,
numberstyle=\tiny,
numbersep=5pt,
breaklines=true,
showstringspaces=false,
basicstyle=\footnotesize,
emph={label}}

\section{How to Collect Data or Get Pre-Existing Data}


  \begin{enumerate}
  \item Data sources

  \item Building Data Collection %(Dave and Kosta)
  
    \begin{itemize}
    \item
      What data is relevant to collect? (data granularity depends on needs of subsequent analysis and anticipated space and time overhead; collecting aggregate data vs. individual events; mention that data should be timestamped as a lot of subsequent analyses depend on that)
    \item
      How to modify existing tools to collect data? (events are a good communication paradigm; centralizing the logger in the code allows for consistent log messages and easy maintenance; trade-offs in storing logs as highly structured or somewhat unstructured data)
    \item
      Privacy and anonymity considerations when collecting data. (what not to collect, using hashes to protect user identities)
    \end{itemize}

  \item Instrumenting the IDE %(Will \& ?)

\begin{enumerate}
	\item 
	Identify the goal and metrics for measurement that requires IDE data.  express attributes of the goal such as how time should be measured, how events should be classified or grouped, and derived calculations.
	\item
	Identifying  IDE API interfaces necessary to accomplish the measurement goal.  Examples window showing events, scrolling in the editor, clicking on code lines, Command events, automated events (e.g. build begin and end).
	\item
	The event interceptor model, intercepting IDE events transparently to the user.  consider Throttling the capture of near real-time events such as scrolling.
	\item
	Considerations for anonymizing the collected data
	\item
	IDE data collection tooling.  Build tooling for ease of use and installation.  Planning logging interface for offline and online data collection.  Handling data store access.  Ways to address security.  
	\item
	End user communication including anonymity, statement of intended use, restrictions on use, optional participation
	
\end{enumerate}

  \end{enumerate}

\section{ How to Extract Attributes from Usage Data}


  \begin{enumerate}
  \item Creating meaningful classification of events
	\begin{enumerate}
	\item
	Establish a goal for categorization
	\item
	Create categories for the goal
	\end{enumerate}
  \item Defining useful and balanced sequences % (Will, Anil)
	\begin{enumerate}
	\item
	Define a goal for sequencing data
	\item
	Sequences aligned with time or time events.  Eg. events per day or events
	between periods of inactivity.
	\item
	Natural sequences such as events between builds, events between edits, events between check-ins.
	\item
	Creating sequences that occur along an event count boundary
	\item 
	Dealing with multi-event dependent sequences where the sequence may depend on detection of a cluster of events
	\end{enumerate}
  \item  Modeling sequences with state models %(Anil)
                \begin{enumerate}
                \item Convert sequence data to usage data
                \item Calculate usage probability
                \item Calculate transitional probability
                \item Visualization
                \item Data mining
                \end{enumerate}


  \end{enumerate}
\section{ Perspectives on Data Analysis}


\begin{enumerate}
\item Creating and communicating developer oriented feedback
\item Tailoring usage data to provide value to toolsmiths
\item Research perspectives and opportunities
\end{enumerate}
\section{ Reasearch combining usage data with other sources}

  \begin{enumerate}
  \item Tasks 
  \item Change history
  \item Program elements
  \item News feeds
  \end{enumerate}
\section{Limits of What You Can Learn from Usage Data}\label{sec:limitations}

Collecting usage data is potentially a boon to research, and as we have
pointed out, has many interesting and impactful applications.
Nonetheless, there are limits to what you can learn as a researcher
from usage data.
In fact, our experience is that researchers (and practitioners) have 
high expectations about what they can learn from usage data, and those
expecations often come crashing down after significant effort implementing
and deploying a data collection sytem.
So before you begin your usage data collection and analysis, consider
the following two limitations of usage data.

\textbf{Rationale is Hard to Capture.}
Usage data tells you what a software developer did, but not
why she did it.
For example, if your usage data tells you that a developer used 
new refactoring tool for the first time, from a trace alone you cannot determine whether
(a) she learned about the tool for the first time, (b) she had used the tool before,
but before you started collecting data, or (c) her finger slipped and 
she pressed a hotkey by accident.
A researcher may be able to distinguish between these by collecting additional information,
like asking the developer just after she used the tool why she used it,
but it is impossible to definitively separate these based on the 
usage data alone.

\textbf{Usage Data Never Captures Everything.}
Researchers often want to capture ``everything,'' or at least 
everything that a developer does.
This is fundamentally impossible, and the reason we introduced the 
goal-question metric.
If you have a system that captures all key presses, you are still 
lacking information about mouse movements.
If you have a system that also captures mouse movements, you're still
missing the files that the developer is working with.
If your system captures also the files, you still lacking the 
histories of those files.
And so on.
Ultimately, usage data is all about fitness for purpose -- is the data
you are analyzing fit for the purpose of your research questions.

To avoid these limitations, we recommend not thinking about 
usage data collection abstractly, but thinking about it concretely before
you begin.
Invent an ideal usage data trace, and ask yourself:

\begin{itemize}
  \item Does the data support my hypothesis?
  \item Are there alternative hypotheses that the data would support?
  \item Can the data be feasibly generated?
\end{itemize}

\noindent
Answering these questions will help you determine whether you can sidestep
the limitations of collecting and analyzing usage data.
\section{A Look Ahead}

  \begin{enumerate}
  \item - are we creating Big Brother? risks for unintended use
  \item - long term effects of developer monitoring 
  \item - 
  \end{enumerate}

\pagebreak
%\section*{Table of Contents}
%\begin{enumerate}
%  \item What Is Usage Data and Why Analyze It?
%  \item How to Collect Data or Get Pre-Existing Data
%  \begin{enumerate}
%  \item Data sources
%  \item Building Data Collection %(Dave and Kosta)
%  \item Instrumenting the IDE %(Will \& ?)
%  \item Video studies %(?)
%  \item Other methods
%  \end{enumerate}
%  \item How to Extract Attributes from Usage Data
%  \begin{enumerate}
%  \item Creating meaningful classification of events
%  \item Defining useful and balanced sequences % (Will, Anil)
%  \item Dealing with multi-event dependent sequences
%  \item  Modeling sequences with state models %(Anil)
%  \end{enumerate}
%\item{Perspectives on Data Analysis}
%\begin{enumerate}
%\item Creating and communicating developer oriented feedback
%\item Tailoring usage data to provide value to toolsmiths
%\item Research perspectives and opportunities
%\end{enumerate}
%\item Reasearch combining usage data with other sources 
%  \begin{enumerate}
%  \item Tasks 
%  \item Change history
%  \item Program elements
%  \item News feeds
%  \end{enumerate}
%  \item Limits of What You Can Learn from Usage Data 
%  \item A Look Ahead
%\end{enumerate}

  
\bibliographystyle{abbrv}
\bibliography{paper_id1} 

\end{document}

