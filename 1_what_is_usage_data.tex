\section{Usage Data Research Concepts}

\subsection{What Is Usage Data and Why Analyze It?}

We refer to the data about the interactions of users with an IDE as the IDE's
usage data or simply \emph{usage data}.

% Who are the stakeholders in this domain?
%
% How do the stakeholders benefit from usage data?
Several stakeholders benefit from capturing and analyzing usage data. First, IDE
vendors leverage the data to get insight about ways to improve their product
based on how their users use the IDE in practice. Second, researchers both
develop usage data collectors and conduct rigorous experiments to (1)~make
broader contributions to our understanding of programmers' coding practices and
(2)~improve the state-of-the-art productivity programming tools. Finally,
programmers benefit from the analysis conducted on the usage data because these
analyses lead to more effective IDEs that make programmers more productive.

% What are some examples of usage data?
Abstractly, an IDE can be modeled as a complex state machine. In this model, a
user performs an action at each step that moves the IDE from one state to
another.
%
% The following is related to the subsection on "Usage Data Never Captures
% Everything".
Perhaps a video recording of the IDE accompanied by the keyboard strokes and
mouse clicks of the user would provide a complete set of usage data.
%
While this usage data in multimedia format may be suitable for small lab
studies, it suffers from two major limitations. First, it is difficult to
automatically analyze usage data in this format. Second, the storage and
collection of usage data in this format is inefficient.

% Give examples of usage data.
To enable experiments beyond small labs, researchers and IDE vendors have
developed various usage data collectors (\SecRef{SecHowToCollectData}).
Depending on the goals of the experiments, the usage data collector captures a
subset of the IDE's state machine.
%
Mylar~\cite{V:Murphy2006How} was one of the first usage data collectors. Mylar
was implemented as a plug-in for the Eclipse IDE. It captured users' navigation
histories and their command invocations. For example, Mylar recorded changes to
selections, views, perspectives as well as invocations of commands such as
delete, copy, and automated refactorings.

Later, Eclipse incorporated a system similar to Mylar, called the Eclipse Usage
Data Collector (UDC), as part of its standard distribution package for several
years.

	-tool usage
	-resource usage (file/method/etc)
	-click streams

need for analysis
	-answers questions about how developers work
	-example questions in the past e.g. refactoring
	-who uses the data
		-researchers
		-toolsmiths
		-users (?)

benefits
	-understanding that you don't get from lab experiments. more realistic data
	-you can also use it in the lab, but screencast is also inefficent in storage and analysis
		- can get people working on their own tasks and own code with more realistic data
		- can use in lab to make analysis easier than screencast
		- longer period of time



\subsection{Design of Usage Data Collection}


\subsubsection{Selecting Relevant Data Based on a Goal}
	Tailoring usage data collection to specific needs helps optimize the volume of data and privacy concerns when collecting information from software development applications.  While a general solutions described in the next sections collect all events from the Integrated Development Environment (IDE), limiting the data collection to specific areas for research can improve the result.  A process for defining the desired data can follow structures such as Goal-Question-Metric \cite{basili-GQM}  that refines a high-level goal into specific metrics to generate from data.  For example, to study the navigation practices of developers, we perfrom the GQM process as follows:
    \begin{itemize}
\item
	Goal
\subitem
	To assess and compare the use of structured navigation by developers in our study
\item
	Possible Question(s)
\subitem
	What is the frequency  of navigation commands developers use when modifying source code?
\subitem
	What portion of navigation commands developers use are structured navigation rather than unstructured navigation?
\item
	Metric
\subitem
	Navigation Ratio is the proportion of the number of structured navigation commands used to the number of unstructured navigation commands used by a developer in a given time period (e.g. a day)

	    \end{itemize}

The specific way to measure navigatoin ration from usage data needsfurther refinement to determine how the usage monitor can identify these actions from available events in the IDE. This is discussed in further detail in section \ref{buildItYourself}.

Assessing commands within a time duration (e.g. day) requires that we collect a time-stamp for each command.  Simply using the time as a "by variable" where we stratify data according to time is a straight-forward conversion.  The time-stamp can be converted to a date allowing data to be grouped by the day of the events.  Similarly the time-stamp can be convered to hour to look at events each hour of any given day.
Calculating duration or elapsed time for a command or set of commands adds new requirements to monitoring.  Specifically, the need to collect events that relate to when the application or IDE is being used and when it is not and the need to collect additional events that occur when the user has moved on from the command of interest.  


\subsubsection{Privacy Concerns}

Concerns over privacy makes usage monitoring a flash point for some people.  Privacy concerns can be divided into concerns about what data is collected that may expose the user or parts of the source code to prying eyes and who the data and aggregates of it are shared with.

To alleviate concerns about what data is collected, steps such as hashing sensitive pieces of information can reduce the concern. If you collect information like window titles they can contain filenames, web site titles, or even email titles.  Hashing these names provides a measure of assurance that the data are less easily identifiable with the user of the system and with sensitive subjects they may be working on.   Collecting a user identification makes this data very helpful for understanding distribution of practices, however, this makes it more sensitive as well.  Generating an anonymous identifier to identify distinct users helps alleviate the concern that data will be associated with the user.

To reduce concerns about who the data are shared with, create a statement of policy around this.  If the data is for research purposes only, you can state specifically whom will have access to the data and what they will do with it.  Limiting statements such as not reporting data at the individual level helps this policy reduce concerns about who the data is shared with.


\subsection{ Data sources}
    \begin{itemize}
    \item
	- usage data from tools of course
\item
	- usage data in user study smaller setting is augmented by user explicit feedback to establish ground truth (e.g. search result was good)
\item
	- smaller studies can ask questions about user activities that add rationale to the usage data that tells why user is doing what they are doing
\item
	- small studies can generate a metric you can use with larger data source that does not have augmented data.  E.g. structured navigation that Robillard did supports a larger study such as what we did at ABB.
    \end{itemize}
    
