\section{How to Collect Data or Get Pre-Existing Data}


  \begin{enumerate}
  \item Data sources
    \begin{itemize}
    \item
	- usage data from tools of course
\item
	- usage data in user study smaller setting is augmented by user explicit feedback to establish ground truth (e.g. search result was good) 
\item
	- smaller studies can ask questions about user activities that add rationale to the usage data that tells why user is doing what they are doing
\item 
	- small studies can generate a metric you can use with larger data source that does not have augmented data.  E.g. structured navigation that Robillard did supports a larger study such as what we did at ABB.
    \end{itemize}

  \item Building Data Collection %(Dave and Kosta)
  
    \begin{itemize}
    \item
      What data is relevant to collect? (data granularity depends on needs of subsequent analysis and anticipated space and time overhead; collecting aggregate data vs. individual events; mention that data should be timestamped as a lot of subsequent analyses depend on that)
    \item
      How to modify existing tools to collect data? (events are a good communication paradigm; centralizing the logger in the code allows for consistent log messages and easy maintenance; trade-offs in storing logs as highly structured or somewhat unstructured data)
    \item
      Privacy and anonymity considerations when collecting data. (what not to collect, using hashes to protect user identities)
      
     
    \item
    	Existing usage data (Mylyn's 99 developers, UDC, etc). %Emerson 
     
    \end{itemize}

  \item Instrumenting the IDE -- Using an Existing Framework %Emerson
  
\begin{enumerate}
	\item 
	There are many existing frameworks to use. We suggest picking on and building off one. (Note: Clone these on github? include packaged with IDE?)
	
	\item Eclipse: Mylyn Monitor. Format: What it was originally intended to be used for. What data it collects. Where to get it. How to install it (w/version info). Example output. Deployment(?)
	
	\item CodingSpectator %https://github.com/vazexqi/CodingSpectator	
	
	\item Hackystat\cite{V:johnson2003beyond}, %https://code.google.com/p/hackystat/

	\item Zorro\cite{Kou2010Operational}. %same as Hackystat? How about VS tool?
	
	\item CodeAlike has one, too, but they get the data!
	
	\item Romain Robbes -- maybe for Visual Studio, but for smalltalk? SHould be in chapter.
\end{enumerate}

  \item Instrumenting the IDE -- Building it Yourself for Visual Studio %(Will \& ?)
 summary:
\begin{enumerate}
	\item 
	Identify the goal (or research questions) and metrics for measurement that requires IDE data.  express attributes of the goal such as how time should be measured, how events should be classified or grouped, and derived calculations.
	\item
	Identifying  IDE API interfaces necessary to accomplish the measurement goal.  Examples window showing events, scrolling in the editor, clicking on code lines, Command events, automated events (e.g. build begin and end).
	\item
	The event interceptor model, intercepting IDE events transparently to the user.  consider Throttling the capture of near real-time events such as scrolling.
	\item
	Considerations for anonymizing the collected data
	\item
	IDE data collection tooling.  Build tooling for ease of use and installation.  Planning logging interface for offline and online data collection.  Handling data store access.  Ways to address security.  
	\item
	End user communication including anonymity, statement of intended use, restrictions on use, optional participation
	
\end{enumerate}
- detailed how to walkthrough 
  \end{enumerate}

\subsection{Eclipse Usage Data Collector}

This section outlines how to collect IDE usage data using Eclipse's Usage Data Collector (UDC)(\url{http://www.eclipse.org/epp/usagedata/}).
The UDC framework was originally build by the Eclipse Foundation, as a way to measure how the 
community was using the Eclipse IDE.
While UDC was included in official Eclipse releases and data was collected from
hundreds of thousands of Eclipse users between 2008 and 2011, the project was eventually shut down, 
and UDC was removed from official Eclipse releases.
However, the source code for UDC remains available and researchers can still use and deploy it.

\subsubsection{What Data is Collected}

The Eclipse Data Collector records the following types of Eclipse information:

\begin{itemize}
 
\item Information about the runtime environment, such as the operating system and Java Virtual Machine.

\item Information about the Eclipse environment, such as which bundles are loaded and when Eclipse
starts up and shuts down

\item Information about actions and commands that are executed, via menus, buttons, toolbars, and hotkeys.

\item Information about when views, editors, and perspectives are invoked.

\end{itemize}

\noindent

Let's look at an example of an event that UDC produces on a developer's machine:
\vspace{4mm}

\begin{tabular}{llllll}
\textbf{what}&\textbf{kind}&\textbf{bundleId}&\textbf{bundleVersion}&\textbf{description}&\textbf{time}\\
executed&command&org.eclipse.ui&3.7.0.v20110928-1505&org.eclipse.ui.edit.paste&1389111843130\\
\end{tabular}

\vspace{4mm}
The first column tells us what kind of thing happened -- in this case, something was executed.
The second column tells us what was executed -- in this case, a command.
The third column tells us what bundle this event belonged to -- in this case, Eclipse's user interface bundle.
The fourth column gives us the version of the bundle.
The fifth tells us the name of the command that was executed -- in this case, paste.
The final column tells us when the command was executed, in Greenwich Mean Time -- in this case, January 7th, 2014 at 16:24:03 GMT.


\subsubsection{Limitations}

Apart from general limitations of collecting usage data~\ref{sec:limitations},
one significant limitation of UDC that we have found is that sometimes it has 
unexpectedly incomplete data.
For example, in planning for a study involving when people ran their JUnit tests,
we found that UDC recorded an event when the ``Run>Run As>Run as JUnit Test'' menu item was selected,
but not when the ``Run As'' button was pressed on the toolbar.
We suspect that the reason has to do with how different user interface affordances 
invoke the same functionality.
In general, when you are planning on running a study with UDC, be sure to know what 
types of events you are looking for, and test them to make sure UDC captures those events.

\subsection{How to Use It}

Collecting data for your own research is fairly straightforward with Eclipse UDC,
and we describe how to do so here.
We also include an accompanying screencast that shows the basics: 
\url{https://docs.google.com/file/d/0B7DV-T4_2mpKRWgwRnJnTWpvN0E/edit}
%TODO move this screencast to youtube, once you're happy

\paragraph{Gathering Data in Experiments using the UDC Client.}

Let's talk about how data is collected on a local machine.
Since UDC was last included in the Eclipse Indigo SR2 release 
(\url{http://www.eclipse.org/downloads/packages/release/indigo/sr2}), 
if you have the option of which Eclipse to use, we recommend downloading
that version.

By default, UDC starts collecting data when Eclipse is start up. 
You can verify this by going to $Windows > Preferences$, then 
select the \textit{Usage Data Collector} item (Figure~\ref{fig:prefPage1}).
The \textit{Enable capture} option should be checked.

First, try out a few commands and views in Eclipse.
Then, on your file system, open the following path as a subdirectory
of your current workspace: \texttt{.metadata/.plugins/org.eclipse.epp.usagedata.recording}
(Figure~\ref{fig:filesystem}).
In that folder, depending on how many UDC events have been gathered 
a number of comma seperated value (CSV) files will appear, where \texttt{upload0.csv} is the oldest
and \texttt{usagedata.csv} is the newest.
Open up \texttt{usagedata.csv} -- you should notice a large number and a variety of events.

Before doing a study, be aware that Eclipse will ask and periodically attempt to upload
data to the Eclipse foundation server.
You should \emph{not} allow it to do this, because each time data is uploaded, the underlying
CSV files are deleted.
Furthermore, because the UDC project is no longer officially supported, the official Eclipse
UDC server no longer accepts the data, so your usage data is, in effect, lost permanently.

Unfortunately, there is no easy way to tell the UDC client to permanently store
usage data.
An easy workaround is to increase the upload period to 90 days (Figure~\ref{fig:upload}),
which should be enough time to complete the experiment.
The long-term fix for this issue is to modify the source code, as we will explain
how to do shortly, to never upload data.

If you're doing a lab experiment, collecting data should be simply a matter of 
copying and deleting the CSV files after each participant has done the experiment.
You can append the files together or put them in a database for analysis.

\paragraph{Modifying the UDC Client.}

\paragraph{Transmitting Data over the Internet}


Alt:
Newer version of Eclipse, integrate from source. (but no guarantees)
If you want to make modifications, checkout also. Use branch associated with release. 
Tried with Indigo R2 + branch.

How to install it (w/version info). 

\subsection{Deployment.}

(Not sure if we want this section. Mostly just point to update sites.)

\subsection{Existing Data}

Maybe where to get some archived data. 