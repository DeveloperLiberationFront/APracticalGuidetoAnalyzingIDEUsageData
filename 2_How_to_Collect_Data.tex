\section{How to Collect Data or Get Pre-Existing Data}


  \begin{enumerate}
  \item Data sources
    \begin{itemize}
    \item
	- usage data from tools of course
\item
	- usage data in user study smaller setting is augmented by user explicit feedback to establish ground truth (e.g. search result was good) 
\item
	- smaller studies can ask questions about user activities that add rationale to the usage data that tells why user is doing what they are doing
\item 
	- small studies can generate a metric you can use with larger data source that does not have augmented data.  E.g. structured navigation that Robillard did supports a larger study such as what we did at ABB.
    \end{itemize}

  \item Building Data Collection %(Dave and Kosta)
  
    \begin{itemize}
    \item
      What data is relevant to collect? (data granularity depends on needs of subsequent analysis and anticipated space and time overhead; collecting aggregate data vs. individual events; mention that data should be timestamped as a lot of subsequent analyses depend on that)
    \item
      How to modify existing tools to collect data? (events are a good communication paradigm; centralizing the logger in the code allows for consistent log messages and easy maintenance; trade-offs in storing logs as highly structured or somewhat unstructured data)
    \item
      Privacy and anonymity considerations when collecting data. (what not to collect, using hashes to protect user identities)
    \end{itemize}

  \item Instrumenting the IDE -- Using an Existing Framework %Emerson
  
\begin{enumerate}
	\item 
	There are many existing frameworks to use. We suggest picking on and building off one. (Note: Clone these on github? include packaged with IDE?)
	
	\item Eclipse: Mylyn Monitor. Format: What it was originally intended to be used for. What data it collects. Where to get it. How to install it (w/version info). Example output. Deployment(?)
	
	\item CodingSpectator %https://github.com/vazexqi/CodingSpectator	
	
	\item Hackystat\cite{V:johnson2003beyond}, %https://code.google.com/p/hackystat/

	\item Zorro\cite{Kou2010Operational}. %same as Hackystat? How about VS tool?

	\item Codealike %https://codealike.org

	\item Romain Robbes univ of chile 	
\end{enumerate}

  \item Instrumenting the IDE -- Building it Yourself for Visual Studio %(Will \& ?)
 summary:
\begin{enumerate}
	\item 
	Identify the goal (or research questions) and metrics for measurement that requires IDE data.  express attributes of the goal such as how time should be measured, how events should be classified or grouped, and derived calculations.
	\item
	Identifying  IDE API interfaces necessary to accomplish the measurement goal.  Examples window showing events, scrolling in the editor, clicking on code lines, Command events, automated events (e.g. build begin and end).
	\item
	The event interceptor model, intercepting IDE events transparently to the user.  consider Throttling the capture of near real-time events such as scrolling.
	\item
	Considerations for anonymizing the collected data
	\item
	IDE data collection tooling.  Build tooling for ease of use and installation.  Planning logging interface for offline and online data collection.  Handling data store access.  Ways to address security.  
	\item
	End user communication including anonymity, statement of intended use, restrictions on use, optional participation
	
\end{enumerate}
- detailed how to walkthrough 
  \end{enumerate}
