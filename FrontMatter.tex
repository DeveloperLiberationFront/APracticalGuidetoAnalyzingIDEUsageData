\begin{center}
\section*{Abstract}
\end{center}
Integrated Development Environments (IDEs) such as Eclipse
and Visual Studio provide tools and capabilities to perform tasks such as navigating among classes and methods, continuous compilation, code refactoring, automated testing, and integrated debugging all designed to increase productivity. 
Instrumenting the IDE to collect usage data gives researchers a greater level of detail on developers' work than previously possible.  Usage data supports analysis of how developers spend their time, what activities might benefit from greater tool support, where developers have difficulty comprehending code, and whether they are following specific practices such as test-driven development.
With usage data, we expect to uncover more nuggets of how developers create mental models, how they investigate code, how they perform mini trial-and-error experiments, and what might drive productivity improvements for everyone.

\section{Introduction}
As software development evolved, Integrated Development Environments (IDEs) sprang up to aid  developers in managing the complexity of software programs.  To increase productivity, modern IDEs such as Eclipse and Visual Studio include tools and capabilities to perform tasks such as navigating among classes and methods, continuous compilation, code refactoring, automated testing, and integrated debugging.  The breadth of development activities supported by the IDE makes collecting editor, command, and tool usage data valuable for analyzing developers' work patterns.  

Prior work on collecting and processing data for recommendation systems describes tools, analysis methods, and important findings from developer usage data analysis~\cite{fritzBookChapter}.  This work is also an excellent introduction to prior research applying usage data to solve issues that challenge developers.  This chapter provides a practical how-to description for collecting and analyzing usage data from IDEs with practical guidance with examples.  

Instrumenting the IDE to collect usage data gives researchers a greater level of detail on developers' work than previously possible. Instrumenting the IDE involves extending the IDE within a provided API framework.  Eclipse and Visual Studio support a rich API framework allowing logging of many commands and actions  as they occur.  Tools used for research that instrument the Visual Studio IDE include Hackystat~\cite{V:johnson2003beyond}, Blaze, and Codealike.  Tools that instrument the Eclipse IDE include Mylyn Monitor, CodingSpectator~\cite{VakilianETAL2012UseDisuseMisuse}, Hackystat~\cite{V:johnson2003beyond}, and Zorro~\cite{Kou2010Operational}.

Usage data supports analysis of how developers spend their time~\cite{V:johnson2003beyond}, developer actions that might benefit from greater tool support~\cite{V:MurphyHill2012How}, where developers have difficulty comprehending code~\cite{Carter2010Are}, even whether they are following specific practices such as test-driven development~\cite{Kou2010Operational}.  Combining usage data with additional data dimensions such as tasks or code change history, researchers can understand larger influences of low-level developer practices.  With these data, we can answer questions such as the level of expertise developers have for an area of source code~\cite{Fritz2010Degreeofknowledge}.

There are limits, however, to what IDE usage data can tell us.  The missing elements include things like  the developer's mental model of the code, or how they intend to alter the code to suit new requirements.  The developers' experience, design ideas, and constraints they keep in mind during an implementation activity are factors that we must obtain separately.  

Looking forward, usage data from development environments provides a platform for greater understanding of low-level developer practices.  We expect to uncover more nuggets of how developers create mental models, how they investigate code, how they perform mini trial and error experiments, and what might release further productivity improvements for all developers.
